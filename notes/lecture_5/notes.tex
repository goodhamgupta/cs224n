\documentclass[a4paper]{article}

\usepackage{amsmath, graphicx, float, blindtext} % for dummy text
\graphicspath{ {./images/} }
\title{Dependency Parsing}
\author{Shubham Gupta}

\begin{document}
\maketitle
\section{Introduction}
\subsection{Phrase structure}
\begin{itemize}
    \item words
    \item phrases
    \item Bigger phrases(sentences lol)
    \item Also called Context Free Grammer(CFG). Building structure for the language using Noun(N), Verb(V), Verb Phrase(VP), etc.
\end{itemize}
\subsection{Dependency structure}
\begin{itemize}
    \item Words depend on other words.
    \item Why do we need all this?
        \begin{itemize}
            \item Understand language structure
            \item Need to know wat is connected is to what.
        \end{itemize}
\end{itemize}

\section{Ambiguities}
\subsection{Prepositional phrase attachment ambiguity}
\begin{itemize}
    \item Occurs when you have PP before a Noun or Noun Phrase or a Verb.
\end{itemize}
\subsection{Coordination phrase ambiguity}
\begin{itemize}
    \item Example: Doctor: No heart, cognitive issues.
\end{itemize}
\subsection{Adjectival Modifier Ambiguity}
\begin{itemize}
    \item Example: Students get first hand job experience(looool)
\end{itemize}
\subsection{Verb Phrase attachment ambiguity}
\begin{itemize}
    \item Example: Mutilated body washes up rio beach to be used for olympics
\end{itemize}
\section{Dependency Grammar}
\begin{itemize}
    \item Syntactic structure between lexical terms (normally arrows called \textit{depedencies}) 
    \item Arrows have the relationship type between the two words. They connect \textit{head} and the \textit{dependant} of the dependency
\end{itemize}
\end{document}
