\documentclass[a4paper]{article}

\usepackage{amsmath, blindtext, float, graphicx, hyperref}
\graphicspath{ {./images/} }
\title{Lecture 16: Coreference Resolution}
\author{Shubham Gupta}

\begin{document}
\maketitle
\section{Introduction}
\begin{itemize}
    \item Identify all mentions that refer to the same real world entity
    \item When one word refers to two or more entities, it is called \textbf{split antecedent}. No system can deal with these words. Eg: A and B went out. \textit{They} are retarded.  
    \item Coref resolution helps in:
    \begin{itemize}
        \item Full text understanding
        \item Machine translation
        \item Dialogue systems
    \end{itemize}
    \item Steps:
    \begin{itemize}
        \item Detect the mentions(easy)
        \item Cluster the mentions(hard) aka coreference
    \end{itemize}
\end{itemize}
\section{Mention Detection}
\begin{itemize}
    \item Span of text referring to some entity
    \begin{itemize}
        \item Pronouns: I, your, it, she him. Use POS tagger
        \item Named entities: people places, Use NER
        \item Noun phrases: a dog, cat stuck in tree. Use parser(constituency parser)
    \end{itemize}
    \item Marking all pronouns, NE and NP over-generates mentions
    \item Solution: Train classifier to filter spurious mentions. 
    \item Solution 2: Collect all mentions as "candidate mentions". Discard mentions that have not been marked as coreference with any other word.
\end{itemize}
\section{Linguistics}
\begin{itemize}
    \item \textit{Anaphora}: One term(anaphor) refers to another term(antecedent). Interpretation for anaphor dependent on interpretation of antecedent
    \item Obama said he would sign the bill.
    \item Obama: Antecedant. he: anaphor
    \item Not all anaphoric relations are coreferential
\end{itemize}
\end{document}
